\setlength{\parindent}{4em}
\setlength{\parskip}{1em}
\renewcommand{\baselinestretch}{2.0}
\section*{Extended Summary (2-4 pages)}
\subsection*{Background}
puzzle important to international migration
\begin{enumerate}
    \item Existing migration theory continues to fall short in explaining current trends in international migration. In order to address this issue, much research has turned  its focus to the migration decision-making process.
\end{enumerate}
summary of existing literature
\begin{enumerate}
	\item Massey et al. 1999 summarize migration theory and say accounting for all the things we think are important we still are not able to really explain migration
	\item "Much thinking on migration remain implicitly or explicitly based on simplistic push–pull models or neo-classical individual income (or ‘utility’) maximising assumptions, despite their manifest inability to explain real-world patterns and processes of migration." (de Haas 2021)
	\item If we want to better understand migration, then we need to know more about what makes people want to migrate
	\item Migration aspirations are predictive of migration
	\item People open to relocation could exhibit behaviors in other ways that do not affect migration (eg less civic engagement, national attachment Marrow et al. 2018); also knowing more about this has other effects that are not migration-related at a population scale (labour/econ/business things).
	\item "Identifying involuntary non-migrants requires survey data. The amount of data has swelled and now include several multi-round multi-country surveys as well as specialized surveys from various parts of the world.[quantmig] However, conceptual and methodological challenges remain. The value of existing data is often hampered by haphazard formulations in survey instruments (Carling and Schewel 2018, Carling 2019). Future empirical work would benefit from a stronger conceptual and methodological foundation.][fume]" (Carling J, M Czaika, and M.B. Erdal (2020) Translating migration theory into empirical propositions. QuantMig Project Deliverable 1.2.)
\end{enumerate}
Theoretical framework
\begin{enumerate}
    \item Focus on two-step approaches (separating desires/intentions/plans from behavior)
    \begin{itemize}
        \item “we use the model as a starting point for consolidating diverse strands of research that disaggregate migration dynamics along similar lines” (Carling and Schewel 2018)
        \item “How might this theoretical framework, which has largely been used to explore the determinants of migration and immobility in the Global South, be applied to study individuals’ (im)mobility decisions in the Global North? Why might immobility, along the spectrum of voluntary to involuntary, be and important question to consider in ‘developed country’ contexts?” (Schewel 2020)
    \end{itemize}
    \item interest in those who want to stay as an active decision "we can define human mobility as people’s capability (freedom) to choose where to live – including the option to stay – rather than the act of moving itself. Moving and staying then become complementary manifestations of the same migratory agency." (de Haas 2021)
    \begin{itemize}
        \item “just as migration must be distinguished from everyday forms of movement, most often by a change in residence for a certain length of time, immobility may be distinguished by continuity in one’s center of gravity, or place of residence, relative to spatial and temporal frames.” (Schewel 2019)
    \end{itemize}
    \item Part of “multi-dimensional inquiry into how people relate to future migration” (Carling, renewing migration debate 2020)
    \item could empirically investigate the 'high positive \& negative liberty' box in Hein de Haas' 2021 framework
\end{enumerate}
The number of surveys inquiring about the migration decision-making process has increased in the last 30 years, enabling researchers to separately analyze the factors influencing the aspiration to migrate and whether that aspiration is converted to behavior (Aslany et al. 2021). [something here on the summary of factors associated w/ aspirations from the lit review?] The insight gleaned from analysis of existing surveys is impressive, however, not without room for improvement. These surveys can be geographically limited (only 7\% of are multi-regional), lack inter-survey comparability (there is much diversity in wording across survey instruments), and suffer from acquiescence bias (many questionnaires ask about preferences for leaving but not staying) (Carling \& Mjelva 2021). In order address this gap, we present a novel digital data source based on LinkedIn users open to work-related relocation.

The availability of digital data sources has exploded due to the growing ubiquity of information and communication technologies (ICTs), which contain the details of how a person interacts with and moves through digital spaces (Cesare et al. 2018). These digital trace data are increasingly being used to estimate demographic processes (Alburez-Gutierrez et al. 2019). Indices on Google searches related to migration intentions, for example, have proved to be a powerful tool to improve predictive modeling of bilateral migration flows (Böhme, Gröger, and Stöhr 2020; Wanner 2020). We present a novel dataset collected via LinkedIn Recruiter, on those open to job-related international relocation. LinkedIn, a professional networking site of over 720 million users, is a key underutilized digital data source, having been used in only one study to date for migration estimation (State et al. 2014). Compared to existing survey data, these data are relatively less expensive to collect, are continuously available, have consistently defined variables across 24 different languages, and provide a global snapshot of openness to migration as recent as the latest update to a person’s LinkedIn profile. Rather than directly asking a person about their migration preferences, people who are actively looking for a job check a box indicating they are "open to relocation" and then are able to add additional locations, thus removing reactionary responses to the survey instrument. Furthermore, we know the condition under which people may decide to move, a feature which Carling \& Mjelva specifically recommend for future data sources in their review. When used to complement existing survey data, this dataset has the potential to offer a more complete picture of migration decisions.

\subsection*{Data, methods, and preliminary results}
As implied by the name, LinkedIn Recruiter enables recruiters to identify potential job candidates. Recruiters can search for users by a variety of qualities including industry, education, and years of experience, to name a few. Of particular interest, is the option to search for users who, after having indicated they are "open to finding a new job", have listed job locations that differ from their current location. We have collected the aggregate number of people open to relocating for a job in a country that differs from the country in which they are currently located for 191 destination countries. This process involves a separate search for each destination country; a single search, for example, would collect the number of LinkedIn users open to relocating to Germany, but not currently in Germany, stratified by their current location, for the top 75 current locations (ranked by number of users). After a preliminary round in July 2020, we have collected this information every two weeks from October 2020 onward, and have constructed a dataset of bilateral flows of prospective job-related relocators. Though the spatial resolution of the current location varies from metropolitan area up to the country, analyses presented here omitted locations below the country-level. Median country-level counts across all dates of data collection have been aggregated to provide a global overview of the size and direction of people open to relocation between regions, showcasing geographic availability and highlighting a large number of those open to relocating within Europe and Eastern Asia as well as from Southern Asia to Western Asia, Europe, and Northern America (Figure 1).

Given the relatively large size of those open to relocate within Europe, we then assessed the relative attractiveness of countries within the European Union, European Free Trade Association, and the United Kingdom through adaptation of a gravity-type model (Cohen et al. 2008). Based on the premise that the number of people who move from one location to another is proportionate to the ratio of the product of the respective populations divided by the distance bewteen the two, we built a linear model to estimate flows against which our observed data could be compared. Our linear model follows the form:
\[log(number open to relocate)_{ij} = β_{0} + β_{1}log(P_{i}) + β_{2}log(P_{j}) + β_{3}log(D_{ij}) + β_{4}log(area_{i}) + β_{5}log(area_{j}) + β_{6}border_{ij} + β_{7}csl_{ij}\]
Where $i$ denotes the current country, $j$ is the prospective destination, $P$ is the number of LinkedIn users, $D$ is the population weighted distance [double check w/ Maciej] between two countries, $area$ is the area in km^2, $border$ is a binary indicator of whether the countries share a land border, and $csl$ is the probability that two people understand a common language. We then identified anamolous destination countries, where the predicted value was much [or some threshold?] higher or lower than what we would expect based on the included geographic and linguistic factors (Figure 2). Of note, XXX and XXX are relatively more popular destinations that would be expected, whereas XXX and XXX are relatively less desirable, after standardizing for geographic and linguistic variables.
% Departures from Cohen et al.:
% \begin{enumerate}
%     \item use the median across dates, instead of keeping all time points. Some country pairs only show up once, despite a 'large' flow (common in countries w/ large population. a more sophisticated future analysis could better handle this (or more data collections).
%     \item only keep destination countries with at least 5\% of their population on LinkedIn. the logic here is that these destinations may not be reliable (but maybe this step is unnecessary with a minimum 'flow' size)
%     \item or a minimum flow size of 25. This is to limit erroneous country pairs, especially since these are not necessarily individual people
%     \item omit rows where other independent variables (e.g. common spoken language) are missing
%     \item 
% \end{enumerate}
% \subsection*{Expected findings}
% \begin{enumerate}
%     \item table of observations (pairs) over time (maybe something in the appendix with complete list of countries)
%     \item chord (circle sankey) diagram; panel A: global by region panel B: EU+ countries
%     \item rank of top destinations (number, percent) compared with MAFE, GWP, etc.
%     \item CV heat map of panel A: EU+ countries; panel B: some other important groups/zooming in? maybe HICs?
%     \item line plot of divergent trends over time (e.g. Poland -> Belarus/Ukraine/Hungary)
%     \item gravity-type model results of relative rankings
%     \item comparing potential vs. realized migration ("unmet need")
% \end{enumerate}
\subsection*{Conclusion}
Our initial results describe the global, unadjusted regional patterns amongst LinkedIn users who are open to relocate for a new job. Furthermore, we evaluate the relative desirability of destination countries within a European context, [something meaningful and interesting here]. These analyses begin to address how this dataset could be used to complement existing research on the migration decision-making process. Despite the niche that these data can fill, there are a number of challenges. First, the population represented on LinkedIn is not representative of the general population and will be further biased by differential internet access and usage of LinkedIn across countries. It is not possible to directly collect age or gender-stratified data from the recruiter platform, to avoid discriminatory hiring practices, posing an additional challenge. Additionally, since we can only the top 75 current locations per prospective destination, the location pairs are biased towards countries with large populations of LinkedIn users. It is also possible that companies with a propensity to recruit via LinkedIn are clustered in specific locations, allowing for the possibility of job-seekers joining LinkedIn specifically to find a job in those locations. Lastly, without further data collection we do not know the migration history of this population nor where they are on the spectrum from relocation being a distant future, to having already made concrete plans.