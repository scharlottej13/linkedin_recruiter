\setlength{\parindent}{4em}
\setlength{\parskip}{1em}
\renewcommand{\baselinestretch}{2.0}
\section*{Extended Summary (2-4 pages)}
\subsection*{Background}
puzzle important to international migration
\begin{enumerate}
    \item Existing migration theory continues to fall short in explaining extant trends in international migration. In order to address this issue, much research has turned  its focus to the migration decision-making process.
\end{enumerate}
summary of existing literature
\begin{enumerate}
	\item Massey et al. 1999 summarize migration theory and say accounting for all the things we think are important we still are not able to really explain migration
	\item "Much thinking on migration remain implicitly or explicitly based on simplistic push–pull models or neo-classical individual income (or ‘utility’) maximising assumptions, despite their manifest inability to explain real-world patterns and processes of migration." (de Haas 2021)
	\item If we want to better understand migration, then we need to know more about what makes people want to migrate
	\item Migration aspirations are predictive of migration
	\item People open to relocation could exhibit behaviors in other ways that do not affect migration (eg less civic engagement, national attachment Marrow et al. 2018); also knowing more about this has other effects that are not migration-related at a population scale (labour/econ/business things).
	\item "Identifying involuntary non-migrants requires survey data. The amount of data has swelled and now include several multi-round multi-country surveys as well as specialized surveys from various parts of the world.[quantmig] However, conceptual and methodological challenges remain. The value of existing data is often hampered by haphazard formulations in survey instruments (Carling and Schewel 2018, Carling 2019). Future empirical work would benefit from a stronger conceptual and methodological foundation.][fume]" (Carling J, M Czaika, and M.B. Erdal (2020) Translating migration theory into empirical propositions. QuantMig Project Deliverable 1.2.)
\end{enumerate}
Theoretical framework
\begin{enumerate}
    \item Focus on two-step approaches (separating desires/intentions/plans from behavior)
    \begin{itemize}
        \item “we use the model as a starting point for consolidating diverse strands of research that disaggregate migration dynamics along similar lines” (Carling and Schewel 2018)
        \item “How might this theoretical framework, which has largely been used to explore the determinants of migration and immobility in the Global South, be applied to study individuals’ (im)mobility decisions in the Global North? Why might immobility, along the spectrum of voluntary to involuntary, be and important question to consider in ‘developed country’ contexts?” (Schewel 2020)
    \end{itemize}
    \item interest in those who want to stay as an active decision "we can define human mobility as people’s capability (freedom) to choose where to live – including the option to stay – rather than the act of moving itself. Moving and staying then become complementary manifestations of the same migratory agency." (de Haas 2021)
    \begin{itemize}
        \item “just as migration must be distinguished from everyday forms of movement, most often by a change in residence for a certain length of time, immobility may be distinguished by continuity in one’s center of gravity, or place of residence, relative to spatial and temporal frames.” (Schewel 2019)
    \end{itemize}
    \item Part of “multi-dimensional inquiry into how people relate to future migration” (Carling, renewing migration debate 2020)
    \item could empirically investigate the 'high positive \& negative liberty' box in Hein de Haas' 2021 framework
\end{enumerate}
The number of surveys reporting on the determinants of the migration decision-making process has increased in the last 30 years, enabling researchers to answer questions related not only to those who have already migrated, but also on the factors influencing both the formation of migration aspirations and whether those aspirations are converted to mobility (Aslany et al. 2021). The insight gleaned from analysis of existing surveys is impressive, however, not without room for improvement. Carling & Mjelva's recent review reveals these surveys can be geographically limited (only 7\% of are multi-regional), lack inter-survey comparability (there is much diversity in wording across survey instruments), and suffer from acquiescence bias (many questionnaires ask about preferences for leaving but not staying) (Carling & Mjelva 2021).

The availability of digital data sources has exploded due to the growing ubiquity of information and communication technologies (ICTs), which contain the details of how a person interacts with and moves through digital spaces (Cesare et al. 2018). These digital trace data are increasingly being used to estimate demographic processes (Alburez-Gutierrez et al. 2019). Indices on Google searches related to migration intentions, for example, have proved to be a powerful tool to improve predictive modeling of bilateral migration flows (Böhme, Gröger, and Stöhr 2020; Wanner 2020). We present a novel data set of LinkedIn users, collected via LinkedIn Recruiter, on those open to job-related international relocation. LinkedIn, a professional networking site of over 720 million users, is a key underutilized digital data source, and has been used in only one study to date for migration estimation (State et al. 2014). Compared to existing survey data, the data collected via LinkedIn are relatively less expensive to collect, have consistently defined variables across locations, and provide a global snapshot of openness to migration as recent as the latest update to a person’s LinkedIn profile. Rather than directly asking a person about their migration preferences, people add locations to which they're open to relocating for a new job, thus removing reactionary responses to the survey instrument. When used to complement existing survey data, this dataset has the potential to offer a more complete picture of migration decisions.

% \subsubsection*{main benefits of this dataset}
% \begin{enumerate}
%     \item many surveys phrase questions with an implicit assumption that respondents prefer to stay (that's not great because reasons). in this data set we capture all the people who during their job search process express an interest in international relocation (and that's better because reasons)
%     \item these data are more recent (though w/ the caveat that people could have not gone back and updated their profiles once they found a job)
global approach, continuous availability of data, high spatial resolution (metropolitan area?), common definitions of all variables across countries <<<< NEED TO INCORPORATE THIS LINE
% \end{enumerate}

\subsection*{Data}
We have collected data from LinkedIn Recruiter on the aggregate number of people open to relocating for a job in a prospective country that differs from the country in which they are currently located. The dataset is constructed by executing searches for each destination country of interest. A single search, for example, would collect the number of people LinkedIn users open to relocating to Germany, but not currently in Germany, stratified by their current location, for the top 75 results (ranked by number of users). After a preliminary round in July 2020, we have collected this information every two weeks from October 2020 onward for [XX number] prospective destination countries, therefore constructing a dataset of bilateral flows of prospective job-related relocators. It is worth noting the spatial resolution of the current location varies from metropolitan area up to a country, and initial analyses were based only on locations reported at the country-level.

\subsection*{Methods}
The first objective of this project is to explore the utility and limitations of this novel dataset. This will involve statistical modeling to determine if observed trends in migration desire are explained by commonly used techniques for modeling migration, such as a gravity model (Wright and Ellis 2016) and other adaptations (Cohen et al. 2008). The next step would be to compare these data to other digital and survey data sources to assess plausibility. The results of the initial, primarily descriptive, analysis will drive forward further exploration into the factors influencing migration aspirations.

Departures from Cohen model, the decision \& rationale:
\begin{enumerate}
    \item use the median across dates, instead of keeping all time points. Some country pairs only show up once, despite a 'large' flow (common in countries w/ large population. a more sophisticated future analysis could better handle this (or more data collections).
    \item only keep destination countries with at least 5\% of their population on LinkedIn. the logic here is... hard to justify. but practically, I had too many predictors and not enough observations, and chose to limit by destination b/c that's how the data are collected in the first place (seems less likely to introduce other spurious bias that way?)
    \item or a minimum flow size of 25. This is to limit erroneous country pairs, especially since these are not necessarily individual people
    \item omit rows where other independent variables (e.g. common spoken language) are missing
    \item 
\end{enumerate}

\subsection*{Expected findings}
\begin{enumerate}
    \item table of observations (pairs) over time (maybe something in the appendix with complete list of countries)
    \item chord (circle sankey) diagram; panel A: global by region panel B: EU+ countries
    \item rank of top destinations (number, percent) compared with MAFE, GWP, etc.
    \item CV heat map of panel A: EU+ countries; panel B: some other important groups/zooming in? maybe HICs?
    \item line plot of divergent trends over time (e.g. Poland -> Belarus/Ukraine/Hungary)
    \item gravity-type model results of relative rankings
    \item comparing potential vs. realized migration ("unmet need")
\end{enumerate}

\subsection*{Discussion}
\subsubsection*{limitations}
\begin{enumerate}
    \item we do not know where these people are on the spectrum of idea -> action, i.e. they could be actively planning an international relocation or scoping out possibilities, or daydreaming about a distant, imagined future (we hope to investigate this in future work w/ a survey, further data collection, etc.)
    \item maybe these people are not active job-seekers? how many people never updated their profile again after saying 'open to: x?
    \item bias in demographics represented, they are not the general population; bias towards countries w/ companies more likely to recruit via LinkedIn (i.e. people sign up to get a job in those countries)
    \item don't know the migration history, family history, and a bunch of other things that are important (again, this could be turned into a 'future directions' bit)
    \item only 75 top countries of origin
    \item different Linkedin penetration
\end{enumerate}
%including future work 