"To efficiently monitor migration flows, defined as the arrival in the country of persons
of a foreign origin, it is helpful to have quick access to information on the trends and
characteristics of current flows (Willekens et al. 2016)." (wanner et al)
summary of existing literature
\begin{enumerate}
	\item Massey et al. 1999 summarize migration theory and say accounting for all the things we think are important we still are not able to really explain migration
	\item "Much thinking on migration remain implicitly or explicitly based on simplistic push–pull models or neo-classical individual income (or ‘utility’) maximising assumptions, despite their manifest inability to explain real-world patterns and processes of migration." (de Haas 2021)
	\item If we want to better understand migration, then we need to know more about what makes people want to migrate
	\item Migration aspirations are predictive of migration
	\item People open to relocation could exhibit behaviors in other ways that do not affect migration (eg less civic engagement, national attachment Marrow et al. 2018); also knowing more about this has other effects that are not migration-related at a population scale (labour/econ/business things).
	\item "Identifying involuntary non-migrants requires survey data. The amount of data has swelled and now include several multi-round multi-country surveys as well as specialized surveys from various parts of the world.[quantmig] However, conceptual and methodological challenges remain. The value of existing data is often hampered by haphazard formulations in survey instruments (Carling and Schewel 2018, Carling 2019). Future empirical work would benefit from a stronger conceptual and methodological foundation.][fume]" (Carling J, M Czaika, and M.B. Erdal (2020) Translating migration theory into empirical propositions. QuantMig Project Deliverable 1.2.)
\end{enumerate}
Theoretical framework
\begin{enumerate}
    \item Focus on two-step approaches (separating desires/intentions/plans from behavior)
    \begin{itemize}
        \item “we use the model as a starting point for consolidating diverse strands of research that disaggregate migration dynamics along similar lines” (Carling and Schewel 2018)
        \item “How might this theoretical framework, which has largely been used to explore the determinants of migration and immobility in the Global South, be applied to study individuals’ (im)mobility decisions in the Global North? Why might immobility, along the spectrum of voluntary to involuntary, be and important question to consider in ‘developed country’ contexts?” (Schewel 2020)
    \end{itemize}
    \item interest in those who want to stay as an active decision "we can define human mobility as people’s capability (freedom) to choose where to live – including the option to stay – rather than the act of moving itself. Moving and staying then become complementary manifestations of the same migratory agency." (de Haas 2021)
    \begin{itemize}
        \item “just as migration must be distinguished from everyday forms of movement, most often by a change in residence for a certain length of time, immobility may be distinguished by continuity in one’s center of gravity, or place of residence, relative to spatial and temporal frames.” (Schewel 2019)
    \end{itemize}
    \item Part of “multi-dimensional inquiry into how people relate to future migration” (Carling, renewing migration debate 2020)
    \item could empirically investigate the 'high positive \& negative liberty' box in Hein de Haas' 2021 framework
\end{enumerate}