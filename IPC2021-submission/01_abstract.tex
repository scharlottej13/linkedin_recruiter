% Don't count the title or word count in word count
%TC:ignore
\section*{Abstract}
\subsection*{\quickwordcount{01_abstract} / 200 words}
%TC:endignore
WARNING
This part is *very* rough, seems best to write last, but wanted to sketch out a general idea

Lots of good methods (\& data) out there BUT anyone would agree that it’s not enough to capture the liminal space between occurrence of life events and survey data/registers picking up on that. With the advent of digital data, we have an opportunity to study the life events that occur in that space between “it happened” and “was counted”. Previous studies have done loads of cool stuff on using digital data + surveys to advance the field of migration specifically. We have a cool new dataset that also presents this possibility, (describe it briefly), and here we present its flaws alongside opportunities for application in migration decision-making process.
We will walk the reader through a data description of the global dataset, in particular how various countries of destination rank vs other survey data sources (GWP, MAFE, Afrobarometer). This birds-eye view will be followed by a deep dive into Poland as a country of origin, and the “big 5” as countries of destination (US, UK, Canada, UAE, Germany).
The data presented here have the capability to shed light on a difficult group of people to survey, those open to relocation, and can therefore broaden the (something something) help better understand why people decide to move or stay (present opportunity for larger picture of migration, not just people who have already moved).

% 4/23
% feeling stuck b/c I'm not sure what this dataset contributes. which field will care the most about it? How do I fit it into the larger picture without over promising? do I start w/ results and work backwards?

% other ways to frame the 'story' that I have thought of:
% we are in a new era of internetization,  which is shaping (and being shaped by) migration <- I don't love this story b/c I wasn't sure how my data would fit
% migration theory is fragmented and doesn't explain migration very well, so people are using 2-step processes to disentangle the before migration with after migration and this dataset can shed light on this specific group of people <- I like this story the best, because it seems to fit with primarily descriptive results of this dataset
% we don't have good data on migration, and everyone knows this, but there are studies that show intentions can be useful for predicting actual migration <- I think this story would work if I had a prediction model, but I do not
